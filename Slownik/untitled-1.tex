\documentclass[12pt,a4paper,notitlepage]{article}

\usepackage[MeX]{polski}
\usepackage[T1]{fontenc}
\usepackage[utf8]{inputenc}
\usepackage[top=3cm, bottom=2cm, left=3cm, right=3cm]{geometry}
\usepackage{graphicx}

\makeatletter

   \renewcommand\@seccntformat[1]{\csname the#1\endcsname.\quad}
    \renewcommand\numberline[1]{#1.\hskip0.7em}
\newcommand{\linia}{\rule{\linewidth}{0.4mm}}
\renewcommand{\maketitle}{\begin{titlepage}
    \vspace*{2cm}
    \begin{center}\small        
        Instytut Informatyki Uniwersytetu Wrocławskiego\\
        Pracownia Inżynierii Oprogramowania\\
  \vspace{2cm}
        \normalsize \textsc{Grupa 16.}\\
        \normalsize \textsc{\@author}\\
\end{center}
    \vspace{3cm}
    \noindent\linia
    \begin{center}
        \LARGE \textsc{Dokumentacja projektu \textbf{Adminum}}\\       
        \linia
        \vspace{2cm}
        \LARGE \textsc{\@title}\\
      
        \vspace{1.5cm}
       \normalsize \@date\\




    \end{center}
  \end{titlepage}
}
\makeatother

\author{Mirosława Szewczyk, Marek Rybak}
\title{Słownik}

\begin{document}
    \maketitle
\setcounter{page}{2}
    \tableofcontents
    \newpage
    \section{Wstęp}
Niniejszy dokument zawiera spis pojęć użytych w dokumentach dotyczących projektu ,,Adminum".
   \section{Słownik}
\begin{tabular}{lcl}
administrator & -  & osoba zajmująca się zarządzaniem systemem informatycznym i \\
& &odpowiadająca za jego działanie. \\
adres IP & -  & liczba nadawana interfejsowi sieciowemu, grupie interfejsów, bądź \\
& & całej sieci internetowej opartej na protokole IP \\
adres MAC & -  &sprzętowy adres karty sieciowej.\\
baza danych & -  &zbiór danych zapisanych w ściśle określony sposób w strukturach \\
& &odpowiadających założonemu modelowi danych. \\
gniazdko internetowe & -  &złącze stanowiące część instalacji sieciowej, służące do przyłą- \\
& &czania do niej komputerów.  \\
GUI & -  &z ang. \textit{graphical user interface}, graficzny interfejs użytkownika.\\
rekord & -  & zestaw danych, zazwyczaj posiadających ustaloną wewnętrzną\\
& & strukturę, stanowiący pewną całość, ale mogący być częścią\\
& & większego zbioru rekordów. \\
sieć LAN & - & ang. \textit{Local Area Network}, wewnętrzna sieć lokalna




\end{tabular}


	
\end{document}
