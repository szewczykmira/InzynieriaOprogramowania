\documentclass[12pt,a4paper,notitlepage]{article}

\usepackage[MeX]{polski}
\usepackage[T1]{fontenc}
\usepackage[utf8]{inputenc}
\usepackage[top=3cm, bottom=2cm, left=3cm, right=3cm]{geometry}
\usepackage{graphicx}

\makeatletter

   \renewcommand\@seccntformat[1]{\csname the#1\endcsname.\quad}
    \renewcommand\numberline[1]{#1.\hskip0.7em}
\newcommand{\linia}{\rule{\linewidth}{0.4mm}}
\renewcommand{\maketitle}{\begin{titlepage}
    \vspace*{2cm}
    \begin{center}\small        
        Instytut Informatyki Uniwersytetu Wrocławskiego\\
        Pracownia Inżynierii Oprogramowania\\
  \vspace{2cm}
       
      \normalsize \textsc{ Grupa nr 16.}\\
        \normalsize \textsc{\@author}\\
\end{center}
    \vspace{3cm}
    \noindent\linia
    \begin{center}
        \LARGE \textsc{Dokumentacja projektu \textbf{Adminum}}\\       
        \linia
        \vspace{2cm}
        \LARGE \textsc{\@title}\\
      
        \vspace{1.5cm}
       \normalsize \@date\\




    \end{center}
  \end{titlepage}
}
\makeatother

\author{Mirosława Szewczyk, Marek Rybak}
\title{Instrukcja obsługi}

\begin{document}
    \maketitle
\setcounter{page}{2}
    \tableofcontents
    \newpage
    \section{Podstawowe informacje}
	Oprogramowanie ,,Adminum'' zostało napisane aby usprawnić pracę administratorów sieci akademickiej Uniwersytetu Wrocławskiego.
   \subsection{Instalacja aplikacji}
	Aplikacja klienta nie wymaga instalacji, ponieważ jest dostępna przez przeglądarkę internetową przy pierwszym jej uruchomieniu po podłączeniu się do sieci.
  \section{Opis ogólny programu ,,Adminum''}
	Przedsięwzięcie ma na celu zbudowanie aplikacji internetowej wspomagającej zarządzanie siecią komputerową w domach studenckich Uniwersytetu Wrocławskiego. Podstawową funkcją programu będzie zarządzanie bazą danych użytkowników sieci komputerowej domów studenckich. ,,Adminum'' zapewni również narzędzie umożliwiające administratorom sieci zarządzanie dostępem poszczególnych użytkowników do sieci Internet.
    \section{Użytkownicy}
	\subsection{Opis użytkowników}
	Ze względu na zapotrzebowanie na poszczególne funkcje oraz różny zakres uprawnień w sieci komputerowej użytkownicy aplikacji dzielą się na dwie grupy:
	\begin{itemize}
		\item Administratorzy -- osoby zatrudnione przez Uniwersytet Wrocławski do administrowania siecią komputerową domów studenckich.
		\item Mieszkańcy -- osoby korzystające z sieci komputerowej domów studenckich. 
	\end{itemize}
	
	
\end{document}
