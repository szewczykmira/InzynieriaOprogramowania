\documentclass[12pt,a4paper,notitlepage]{article}

\usepackage[MeX]{polski}
\usepackage[T1]{fontenc}
\usepackage[utf8]{inputenc}
\usepackage[top=3cm, bottom=2cm, left=3cm, right=3cm]{geometry}
\usepackage{graphicx}

\makeatletter

   \renewcommand\@seccntformat[1]{\csname the#1\endcsname.\quad}
    \renewcommand\numberline[1]{#1.\hskip0.7em}
\newcommand{\linia}{\rule{\linewidth}{0.4mm}}
\renewcommand{\maketitle}{\begin{titlepage}
    \vspace*{2cm}
    \begin{center}\small        
        Instytut Informatyki Uniwersytetu Wrocławskiego\\
        Pracownia Inżynierii Oprogramowania\\
  \vspace{2cm}
        \normalsize \textsc{Grupa 16.}\\
        \normalsize \textsc{\@author}\\
\end{center}
    \vspace{3cm}
    \noindent\linia
    \begin{center}
        \LARGE \textsc{Dokumentacja projektu \textbf{Adminum}}\\       
        \linia
        \vspace{2cm}
        \LARGE \textsc{\@title}\\
      
        \vspace{1.5cm}
       \normalsize \@date\\




    \end{center}
  \end{titlepage}
}
\makeatother

\author{Mirosława Szewczyk, Marek Rybak}
\title{Analiza ryzyka}

\begin{document}
    \maketitle
\setcounter{page}{2}
    \tableofcontents
    \newpage
 \section{Zarządzanie ryzykiem}
Na postęp w realizacji projektu wpływa wiele czynników, które mogą zahamować proces jego tworzenia. Niniejszy dokument jest poświęcony analizie tego ryzyka oraz sposobach minimalizacji wpływu tych czynników.

	\begin{center} Tabela 1. Analiza ryzyka \end{center}
        \begin{tabular}{|p{3cm}|p{1cm}|p{1cm}|p{1cm}|p{1cm}|p{1cm}|p{1cm}|p{1cm}|p{3cm}|} \hline
            Czynnik ryzyka &
            1\% &
            5\% &
            20\% &
            50\% &
            80\% &
            95\% &
            99\% &
            Sposób minimalizacji ryzyka \\ \hline
            Choroba członka projektu & & & & & & & X & Zapas czasu na realizację zadań \\ \hline
            Choroba wielu pracwników jednocześnie & & & & X & & & & Szczepienia ochronne na koszt firmy \\ \hline
            Poważny błąd w analizie lub projektowaniu wykryty w późnych fazach projektu & & X & & & & & & Konsultacja z pracownikami naukowymi uczelni wyższych \\ \hline
            Nieduże przedłużenie którejkolwiek z faz projektu & & & & & & X & & Zapas czasu na realizację zadań \\ \hline
            Zanczne przedłużenie jednej z faz projektu & & & & X & & & & Przesunięcia pracowników pomiędzy grupami \\ \hline
            Katastrofa, klęski żywiołowe & X & & & & & & & ubezpieczenie, codzienne tworzenie kopii zapasowych \\ \hline
            Wysoka cena usług zwnętrznych & & & & & & X & & Zapas w budżecie \\ \hline
	Brak zaangażowania ze strony zespołu & & & X & & & & & Premie motywacyjne \\ \hline
	Nieznajomość technologii przez członków zespołu & & & & & X & & & Kursy dokształcające przed rozpoczęciem projektu \\ \hline


        \end{tabular}
 


	
\end{document}
