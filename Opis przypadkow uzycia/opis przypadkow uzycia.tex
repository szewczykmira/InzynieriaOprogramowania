\documentclass[12pt,a4paper,notitlepage]{article}

\usepackage[MeX]{polski}
\usepackage[T1]{fontenc}
\usepackage[utf8]{inputenc}
\usepackage[top=3cm, bottom=2cm, left=3cm, right=3cm]{geometry}
\usepackage{graphicx}

\makeatletter

   \renewcommand\@seccntformat[1]{\csname the#1\endcsname.\quad}
    \renewcommand\numberline[1]{#1.\hskip0.7em}
\newcommand{\linia}{\rule{\linewidth}{0.4mm}}
\renewcommand{\maketitle}{\begin{titlepage}
    \vspace*{2cm}
    \begin{center}\small        
        Instytut Informatyki Uniwersytetu Wrocławskiego\\
        Pracownia Inżynierii Oprogramowania\\
  \vspace{2cm}
	\normalsize \textsc{Grupa 16.}\\
        \normalsize \textsc{\@author}\\
\end{center}
    \vspace{3cm}
    \noindent\linia
    \begin{center}
        \LARGE \textsc{Dokumentacja projektu \textbf{Adminum}}\\       
        \linia
        \vspace{2cm}
        \LARGE \textsc{\@title}\\
      
        \vspace{1.5cm}
       \normalsize \@date\\




    \end{center}
  \end{titlepage}
}
\makeatother

\author{Mirosława Szewczyk, Marek Rybak}
\title{Opis przypadków użycia}

\begin{document}
    \maketitle
\setcounter{page}{2}
    \tableofcontents
    \newpage
    \section{Opis przypadków użycia}
Przypadki użycia są realizowane przez dwa rodzaje aktorów:
	\begin{itemize}
	\item użytkownik sieci,
	\item administrator sieci.
	\end{itemize}
	\subsection{Użytkownicy niezalogowani}
Dla niezalogowanych użytkowników oprogramowanie „Adminum” powinno udostępniać dostęp do elektronicznego wniosku rejestracji do sieci, a także dać możliwość zgłoszenia usterek związanych z poprawnym działaniem sieci.
	\subsection{Użytkownicy zalogowani}
		\subsubsection{Sprawdzanie aktualnego stanu w sieci}

Aktor: Użytkownik sieci.\\
Motywacja: Użytkownik chce się dowiedzieć, jaki jest jego aktualny stan w sieci.\\
Scenariusz bazowy: Użytkownik otwiera stronę główną programu „Adminum”, gdzie pojawia się informacja o aktualnym stanie w sieci: zarejestrowany, oczekujący na zaakceptowanie lub odrzucony (wraz z uzasadnieniem).\\
Scenariusz alternatywny: Jeżeli z różnych powodów zgłoszenie użytkownika nie zostało odnotowane w bazie danych, to zostanie on przekierowany na stronę z elektronicznym wnioskiem rejestracji.
		\subsubsection{Zgłaszanie usterek}
Aktor: Użytkownik sieci.\\
Motywacja:  Użytkownik sieci przy pomocy programu „Adminum” ma zamiar zgłosić problem związany z działaniem sieci.\\
Scenariusz bazowy: Użytkownik wybiera zakładkę zawierającą formularz zgłoszeniowy. Można go wypełniać z dowolnego komputera podłączonego do sieci, ponieważ formularz nie pobiera automatycznie adresu karty sieciowej. W blankiecie użytkownik powinien podać informacje o swoim sprzęcie i opisać szczegółowo występujący problem.\\
Scenariusz alternatywny: W przypadku problemu ze stroną zostanie wyświetlony odpowiedni komunikat.
		\subsubsection{Dodawanie użytkowników do sieci}
Aktor: Administrator sieci.\\
Motywacja: Użytkownik chce zaakceptować lub odrzucić zgłoszenia o przyjęcie do sieci.\\
Scenariusz bazowy: Administrator otwiera panel administracyjny przechodzi do zakładki z listą użytkowników oczekujących na zaakceptowanie ich wniosków. Administrator wybiera kandytata z listy i, wchodząc okno z jego danymi, ma możliwość przyjęcia lub odrzucenia jego zgłoszenia, w przypadku odrzucenia podaje powód swojej decyzji.
		\subsubsection{Zarządzanie bazą danych użytkowników}
Aktor: Administrator sieci.\\
Motywacja: Za pomocą programu „Adminum” chce wprowadzić zmiany w danych użytkownika, ustawić zakres możliwości użytkownika lub dodać notatkę do rekordu.\\
Scenariusz bazowy: Użytkownik o uprawnieniach administratora przechodzi w panelu administracyjnym do zakładki z bazą danych użytkowników. Ma tam wgląd do rekordów, które może otworzyć, tzn.  ma dostęp do wszystkich danych użytkownika oraz napisanych do tej pory notatek. Może on zmieniać lub dodawać adnotacje, a także zmieniać zakres możliwości danego użytkownika.\\
Scenariusz alternatywny: Użytkownik z dostępem do panelu administracyjnego, nie posiadający jednak uprawnień administratora, będzie mógł wyłącznie przeglądać rekordy, nie mogąc dokonywać w nich żadnych zmian.	
		\subsubsection{Przeglądanie informacji o usterkach}
Aktor: Administrator sieci.\\
Motywacja: Administrator sieci chce sprawdzić zgłoszone usterki.\\
Scenariusz bazowy: Administrator sieci przechodzi do zakładki ze zgłoszonymi usterkami, gdzie może przeglądać listę usterek opisanych za pomocą krótkich tytułów nadanych im przez osoby zgłaszające. Dopiero po dotarciu do  pełnych danych o rekordzie administrator ma możliwość przeczytania dokładnego opisu problemu oraz odczytania lub poprawienia notateki dodanych przez innych administratorów.\\
Scenariusz alternatywny: Użytkownik z dostępem do panelu sterowania, nie posiadający jednak uprawnień administratora, ma wyłącznie możliwość przeglądania rekordów.
	
\end{document}
